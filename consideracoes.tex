\begin{OnehalfSpace}
    \noindent\textbf{CONSIDERAÇÕES}

    Os estágios proporcionaram diversas novas experiências, como ensino de surdos, e principalmente a regência de turmas do ensino médio.

    Durante as observações, pude notar que os professores possuíam confiança dos alunos, e portanto apesar de parecerem tímidos ou retraídos, os alunos não tinham receio em perguntar, questionar, ou tirar alguma dúvida com os professores.

    A observação me permitiu ter uma visão mais crítica sobre o papel do professor, e adicionar experiências que serão utilizadas quando for exercer o papel de docente no futuro.

    Na regência busquei aplicar as experiências adquiridas na observação, mantendo um estilo próprio e buscando ser sempre solícito com os alunos, ajudando em suas dúvidas e tentando entender as dificuldades de cada um, bem como resolver os problemas que ocorriam em sala de aula, como trabalhos não entregues, faltas, problemas na aplicação de provas em duplas, \textit{et cetera}.

    Acredito que a atual matriz curricular do curso de licenciatura em informática, não prepara corretamente para os desafios que o licenciando irá enfrentar na docência. Segundo Nérici \cite{nerici1987}, a metodologia de ensino é um "conjunto de procedimentos didáticos, representados por seus métodos e técnicas de ensino" para alcançar os objetivos propostos pelo docente. Não há uma profundidade nas matérias técnicas de programação, ou que muitas vezes o ensino de libras, ou docência para pessoas diferentes não é suficiente, levando ao docente, não possuir os métodos e técnicas de ensino.

\end{OnehalfSpace}
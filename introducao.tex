\begin{OnehalfSpace}

    \noindent\textbf{INTRODUÇÃO}

    As salas de aula neste último século, estão mudando. Cada vez mais, a dinâmica entre professor e aluno, vêm se transformando devido ao avanço das tecnologias digitais. Celulares, tablets, redes sociais, vídeos online, bem como a capacidade de a poucos toques numa tela, ter acesso a basicamente qualquer informação têm mudado a dinâmica dos alunos nas salas de aula.
    
    Diante disso, um novo paradigma está surgindo no meio educacional, e o professor, terá um importante papel no ensino e interação com ela, utilizando-se de novas ferramentas e práticas pedagógicas disponíveis graças à essas tecnologias como intercâmbio de dados científicos e culturais, de diversa natureza, produção de texto em língua estrangeira, elaboração de jornais inter-escolas, permitindo o desenvolvimento de ambientes de aprendizagem centrados na atividade dos alunos, na importância da interação social, e no desenvolvimento de um espírito de colaboração e de autonomia nos alunos \cite{machado2002}.
    
    Este documento é um relato de experiência, trazendo informações acerca das atividades desenvolvidas e observações feitas em sala de aula pelo residente, bem como as atividades vivenciadas durante o cumprimento das etapas V e VI da residência pedagógica, realizadas nas datas que compreendem período que vai de setembro de 2019, à janeiro de 2020. 

\end{OnehalfSpace}